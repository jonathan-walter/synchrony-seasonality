\documentclass[letterpaper,11pt]{article}

%packages
\usepackage{amsfonts}
\usepackage{graphicx}
\usepackage[left=2cm,top=2cm,right=2cm,bottom=1.5cm,head=.5cm,foot=.5cm]{geometry}
\usepackage{url}
\usepackage{multirow}
\usepackage{longtable}
\usepackage{subfig}
\usepackage{float}
\usepackage{setspace}
\usepackage{lineno}
\usepackage{natbib}
\usepackage{amsmath}
\usepackage{authblk}
\usepackage{xr}
\usepackage{relsize}
\usepackage{tikz}

%new commands
\newcommand{\ole}{\overline{\epsilon}}
\DeclareMathOperator{\var}{var}
\DeclareMathOperator{\cov}{cov}
\DeclareMathOperator{\cor}{cor}

%header material for paper
\title{Math for Jon and Lauren for seasonality and synchrony paper}
\date{2021 09 21}
\author[a]{Daniel C. Reuman}
\affil[a]{Department of Ecology and Evolutionary Biology and Center for Ecological Research and Kansas Biological Survey, University of Kansas}

%paper starts
\begin{document}

\maketitle

\section{The model}\label{sect:model}

We consider a 2-patch model using $i=1,2$ to index the patches, which are considered identical. The model is
\begin{align}
N_i(t) &= (1-d)N_1(t-1)f[N_1(t-1),\epsilon_{B,1}(t)]s[N_1(t-1)f[N_1(t-1),\epsilon_{B,1}(t)],\epsilon_{W,1}(t)] \label{eq:genmod_1}\\
&+ dN_2(t-1)f[N_2(t-1),\epsilon_{B,2}(t)]s[N_2(t-1)f[N_2(t-1),\epsilon_{B,2}(t)],\epsilon_{W,2}(t)] \\
N_2(t) &= (1-d)N_2(t-1)f[N_2(t-1),\epsilon_{B,2}(t)]s[N_2(t-1)f[N_2(t-1),\epsilon_{B,2}(t)],\epsilon_{W,2}(t)] \\
&+ dN_1(t-1)f[N_1(t-1),\epsilon_{B,1}(t)]s[N_1(t-1)f[N_1(t-1),\epsilon_{B,1}(t)],\epsilon_{W,1}(t)],\label{eq:genmod_4}
\end{align}
where $N_i(t)$ is population density in patch $i$ at time $t$, $0 \leq d < 0.5$ is a
dispersal parameter, the function $f(N,\epsilon_B)$ is a breeding season multiplier,
the function $s(B,\epsilon_W)$ is a winter survival
rate, $\epsilon_{B,i}(t)$ is the environment during the breeding season in patch
$i$ at time $t$, and $\epsilon_{W,i}(t)$ is the environment during the winter in patch
$i$ at time t. We assume $f>0$ and
$\frac{\partial f}{\partial N}<0$, so that more pre-breeding density is always bad
for per-capita population growth during the breeding season. We do not assume $f>1$, so
net death during the breeding season may occur. We assume $\frac{\partial f}{\partial \epsilon_B}>0$, so that higher values of the breeding-season
environmental variable are good for
breeding and breeding-season survival. We assume $0 < s < 1$, and
$\frac{\partial s}{\partial B}<0$, so that more pre-winter, post-breeding-season
individuals are always bad for winter survival. We assume
$\frac{\partial s}{\partial \epsilon_W}>0$, so that higher values of the winter
environmental variable are always good for winter survival. We adopted the
convention that larger values of environmental variables are always ``good'' for
the population. This convention is arbitrary, but is also
adopted without loss of generality, since one
could always replace an environmental variable for which larger values are
``bad'' for the population with its negative. 
For simplicity, we assume the four-dimensional random variables $(\epsilon_{B,1}(t),\epsilon_{B,2}(t),\epsilon_{W,1}(t),\epsilon_{W,2}(t))$
are independent and identically distributed (iid) across time (so the 
$t$ argument can usually be dropped); and
that $(\epsilon_{B,1},\epsilon_{W,1})$ and 
$(\epsilon_{B,2},\epsilon_{W,2})$ are identically distributed and
$(\epsilon_{B,1},\epsilon_{W,2})$ and $(\epsilon_{B,2},\epsilon_{W,1})$
are identically distributed.
The first of these assumptions is a kind of stationarity and white-noise 
assumption, and the second and third assumptions help make our model
spatially homogeneous.

%Jon and Lauren, the explanation above is a bit skeletal, you may want to flesh it
%out, including possibly a flow diagram-type thing showing the steps of the model,
%which are breeding, then winter survival, then dispersal, and the influences
%on the pops at each step. OK, the description should be mathematically complete,
%but you may want to soften it up for an ecological audience by adding a few more
%words and interpretation of the functional forms used and assumptions made.

\section{Linearization}

For this section, we assume $d=0$, for simplicity. We also assume the one-patch,
deterministic version of the model has a positive stable equilibrium, $N^*$, i.e.,
\begin{equation}
N^*=N^*f[N^*,\ole_{B}]s[N^*f[N^*,\ole_B],\ole_W].
\end{equation}
Here, an overline denotes expected value of a random variable. The expected
values $\ole_B$ and $\ole_W$ need no patch index because of assumptions
made about identically distributed environments across patches. We let
$B^*=N^*f[N^*,\ole_B]$, so that $N^*=B^*s[B^*,\ole_W]$. This is the post-breeding,
pre-winter equilibrium density of the one-patch deterministic model. We have
$\frac{B^*}{N^*}=f[N^*,\ole_B]$, $\frac{N^*}{B^*}=s[B^*,\ole_W]$ and
$f[N^*,\ole_B]=\frac{1}{s[B^*,\ole_W]}$. Thus, at average
environmental values and at equilibrium, fecundity and survival balance
each other out, as expected.

We now linearize the no-dispersal model at the equilibrium. Defining
\begin{align}
g(N,\epsilon_B,\epsilon_W) &= N f[N,\epsilon_B] s[N f[N,\epsilon_B],\epsilon_W], \\
n_i(t) &= N_i(t)-N^*, \\
e_{B,i}(t) &= \epsilon_{B,i}(t)-\ole_B, \\
e_{W,i}(t) &= \epsilon_{W,i}(t)-\ole_W,
\end{align}
linearizing gives
\begin{equation}
n_i(t) \approx P_A n_i(t-1) + P_B e_{B,i}(t) + P_W e_{W,i}(t)\label{eq:lin_model}
\end{equation}
for $i=1,2$, where
\begin{align}
P_A &= \left[ \frac{\partial g}{\partial N} \right|_{(N^*,\ole_B,\ole_W)} \\
P_B &= \left[ \frac{\partial g}{\partial \epsilon_B} \right|_{(N^*,\ole_B,\ole_W)} \\
P_W &= \left[ \frac{\partial g}{\partial \epsilon_W} \right|_{(N^*,\ole_B,\ole_W)}.
\end{align}
This is an autoregressive moving average-type model, which can be subjected
to a wide range of mathematical analyses. The stability assumption, above,
is the same as assuming $|P_A|<1$. It is straightforward to show
\begin{equation}
P_A= P_{A,B} P_{A,W},
\end{equation}
where
\begin{equation}
P_{A,B} = \frac{\partial}{\partial N} [Nf(N,\epsilon_B)|_{(N^*,\ole_B)} 
\end{equation}
is the marginal benefit, at equilibrium and at average environment, to the post-breeding population
of a single additional pre-breeding individual, and
\begin{equation}
P_{A,W}= \frac{\partial}{\partial B} [ Bs(B,\epsilon_W) |_{(B^*,\ole_W)} 
\end{equation}
is the marginal benefit, at equilibrium and at average environment, to the post-winter population
of a single additional pre-winter individual. 
The condition $|P_A|<1$ is then equivalent to the condition 
$| P_{A,B} | \times | P_{A,W} | <1$.

\section{Analysis of the linearized model}

To compute synchrony of the model (\ref{eq:lin_model}), we compute 
$\cor(n_i,n_j)$ for $i \neq j$ by computing $\cov(n_i,n_j)$, $\var(n_i)$ and $\var(n_j)$. We have
\begin{align}
\cov(n_i(t),n_j(t)) &= \cov(P_A n_i(t-1) + P_B e_{B,i}(t) + P_W e_{W,i}(t),P_A n_j(t-1) + P_B e_{B,j}(t) + P_W e_{W,j}(t)) \\
&= P_A^2 \cov(n_i(t-1),n_j(t-1)) \\
&+ P_B^2 \cov(\epsilon_{B,i}(t),\epsilon_{B,j}(t)) \\
&+ P_W^2 \cov(\epsilon_{W,i}(t),\epsilon_{W,j}(t)) \\
&+ 2P_B P_W \cov(\epsilon_{B,i}(t),\epsilon_{W,j}(t)).
\end{align}
We here used some of the assumptions about the environmental noise random
variables listed at the end of section \ref{sect:model}. We then have 
\begin{equation}
(1-P_A^2) \cov(n_i,n_j) = P_B^2 \cov(\epsilon_{B,i},\epsilon_{B,j}) 
+ P_W^2 \cov(\epsilon_{W,i},\epsilon_{W,j}) 
+ 2P_B P_W \cov(\epsilon_{B,i},\epsilon_{W,j}).
\end{equation}
Furthermore,
\begin{equation}
\var(n_i(t)) = P_A^2 \var(n_i(t-1)) + P_B^2 \var(\epsilon_{B,i}(t))
+P_W^2 \var(\epsilon_{W,i}(t))+2P_B P_W \cov(\epsilon_{B,i}(t),\epsilon_{W,i}(t)),
\end{equation}
and therefore
\begin{equation}
(1-P_A^2)\var(n_i) = P_B^2 \var(\epsilon_{B,i})
+P_W^2 \var(\epsilon_{W,i})+2P_B P_W \cov(\epsilon_{B,i},\epsilon_{W,i}),
\end{equation}
which holds independently of the value of $i$. Therefore, 
\begin{equation}
\cor(n_i,n_j)=\frac{P_B^2 \cov(\epsilon_{B,i},\epsilon_{B,j})
+ P_W^2 \cov(\epsilon_{W,i},\epsilon_{W,j})
+ 2P_B P_W \cov(\epsilon_{B,i},\epsilon_{W,j})}{P_B^2 \var(\epsilon_{B,i})
+P_W^2 \var(\epsilon_{W,i})+2P_B P_W \cov(\epsilon_{B,i},\epsilon_{W,i})}.
\end{equation}
The correlation $\cor(n_i,n_j)$ approximately equals the correlation
$\cor(N_i,N_j)$ of the original (un-linearized) model, so the main text
should probably show this result:
\begin{equation}
\cor(N_i,N_j) \approx \frac{P_B^2 \cov(\epsilon_{B,i},\epsilon_{B,j})
+ P_W^2 \cov(\epsilon_{W,i},\epsilon_{W,j})
+ 2P_B P_W \cov(\epsilon_{B,i},\epsilon_{W,j})}{P_B^2 \var(\epsilon_{B,i})
+P_W^2 \var(\epsilon_{W,i})+2P_B P_W \cov(\epsilon_{B,i},\epsilon_{W,i})}.
\end{equation}
Note that the first two terms in the numerator relate to traditional
measures of spatial synchrony of environmental variables, but ``cross-variable''
synchrony, both between and within locations (in the numerator and 
denominator, respectively), also contribute. 

There are some interpretations available for the coefficients $P_A$, $P_B$, $P_W$,
but they may or may not be all that interesting, so I'll maybe save those for
another time. Likewise it may (or may not) be interesting to try to develop
interpretations of how cross-site, cross-variable synchrony versus same-site,
cross-variable synchrony may influence population synchrony under 
different circumstances corresponding to different values of $P_B$ and $P_W$.
Also note that the value of $P_A$ does not matter in the end for population
synchrony.

\section{The simulation model}

I think the simulation model should be a special case of the model from 
(\ref{eq:genmod_1})-(\ref{eq:genmod_4}), so it suffices to specify $f$ and $s$.
I think $f(N,\epsilon_B)=\exp(f_0)\exp(-N/K_B)\exp(\epsilon_B)$
is probably a good choice. This is similar to what was used in the version
of the manuscript that I read.
I think $s(B,\epsilon_W)=\min ( \exp(s_0) \exp(-B/K_W) \exp(\epsilon_W),1 )$
is probably a reasonable choice. It might be 
possible to think of a better choice for
$s$, one that does not involve that pesky $\min$ function, but probably not worth 
worrying too much about it. Ricker-like formulations were used previously, but
interpretations of parameters of such models, in this context, as
growth rates and carrying capacities, are not appropriate. Instead we
have $\exp(f_0)$ and $\exp(s_0)$ the fecundity and survival, respectively,
at $0$ density in an average environment, and $K_B$ and $K_W$ 
which control how quickly fecundity and
survival decrease with increasing density. We must require $\exp(s_0) \leq 1$.
There are other constraints on the parameters required for the model
to have a stable equilibrium, it's probably best to just test using simulations
whether the model has a stable equilibrium for whatever parameters you 
choose,and if it does not, choose others. If you end up really needing an 
analytic statement of what parameter constraints have to be imposed to get
a stable equilibrium for this model, let me know.
Note that you should also ensure that the random variables in your simulations
satisfy the assumptions about the random variables listed at the end 
of section \ref{sect:model}. Let me know if those are unclear. 
Your first set of simulations should use $d=0$ and should agree with 
what can be computed from the analytic results above. For that comparison,
you'll need the values of $P_B$ and $P_W$. Let me know if you need help with
that.

\end{document}